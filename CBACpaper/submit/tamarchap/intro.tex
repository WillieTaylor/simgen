\section{Introduction}

Physical methods for the determination of macromolecular structure have
improved greatly over the past decade, resulting in a wealth of large
molecular structures including both protein and nucleic acid.   Most,
if not all of these structures are dynamic, either in their assembly
(and disassembly) such as cytoskeletal fibrils and viruses or often have
large motions associated with their function, such as the F1/F0 ATP
synthetase complex.  Indeed for the class of molecular motors, motion
is the essential aspect of their function \cite{SpudichJA94,AstumianRD01}.

With the exception of real-time scanning atomic force microscopy \cite{KoderaNet10}, which
is still not a widely used method, most of the physical techniques, such
as electron microscopy or X-ray crystallography, provide only snap-shots
of discrete states of a dynamic system \cite{CoureuxPDet04,BershitskySYet09}.   This can be augmented by data
from a wide variety of physical methods, such as spectroscopy or single
particle measurements \cite{ForkeyJNet03,VeigelCet05}, that can provide dynamic information against
which the different 'snap-shots' can be interpreted.   However, given the
coordinates for different states, it is still natural to want to visualise 
how the system progresses from one to the other, or given a set of components,
to wonder how they might assemble into a complex.

An impression of how complex systems behave can be gleaned by interpolating
between states using a physics-based 'morphing engine' \cite{KrebsWGet00}, similar
to those used in animating cartoon characters in popular games and movies. 
Depending on the constraints of the data, these can to some extent be predictive
but more typically, simply recapitulate the relationship between the existing 
structures in a more accessible manner.   At the other extreme to this
top-down approach, molecular dynamics methods can be used to animate large
structures from the bottom up:  that is by summing the interactions of
individual atoms into increasingly larger motions \cite{SchlickT02,BurghardtTPet07,KawakuboTet09}.

Both the top-down and bottom-up approaches have their problems.  The top-down
approach may not have a geometry engine of sufficient quality to ensure that
the system does not wander into un-natural configurations between known
states, which if a problem if there is a large difference between the states
(or just one known state).    Conversely, with the bottom-up approach, the
repeated summation of small atomic forces into large motions is likely to
be a divergent process, giving little confidence in the uniqueness of
the final configuration.

The ideal solution to both these problems is to combine the two approaches.
However, this raises a new set of questions as to the best way that this
can be done and there is a variety of approaches depending on whether the
aim is mainly to predict or recapitulate.   This type of modelling is broadly
referred to as coarse-grained (CG) as the constraints from the data are
imposed at a level above atomic interaction \cite{IzvekovSet05,ZhangZet09,TaylorWRet10a},  
typically in the form of distance-based elastic constraints between neighbouring
units \cite{ParkerDet09,ZhengW11} (being two examples from the actin/myosin motor discussed below).
The definition of levels is essentially arbitrary but for proteins and RNA there is a good
succession in which each level incorporates roughly an order of 10 elements
from the lower level giving a progression form atoms to residues/bases to
secondary structures to domains to multi-domain/chain macromolecules or
assemblies.   This follows the conventional structural hierarchy of
primary, secondary, tertiary and quaternary with the allowance that in the
latter, it is often arbitrary whether the units are linked as domains or
exist as distinct chains \cite{TaylorWRet05}.

As structural constraints are imposed 
at levels above atomic, they becomes less general until the highest
level may be specific to a particular known state of a large system.
For example:  atomic interactions are the same in all molecules, whether
nucleic acid or protein.  Interactions at the monomer level require 
knowledge of the chemical structure of four bases for RNA/DNA and twenty
amino acids for proteins.  At the secondary structure level, both RNA
and protein become very different and the interaction (packing) of the
secondary structures can only be described because of the large numbers
of structures in which they have been observed.  The interaction has
shifted from being fundamental to being empirical.  At the domain level
constraints on the fold of the chain must be derived largely from a
specific fold and at the top level of a multi-domain chain or multi-chain
assembly, constraints derive from a unique structure.

If only one of these levels of constraint is utilised, then the others 
become poorly determined (or must be kept fixed).  As discussed above, using only atomic 
interaction leads to uncertainty at the higher levels and using just 
high-level information (say, domain/domain distances) leaves the
atomic detail unspecified.  Taken by itself, any intermediate level
will leave those above and below less well constrained.  The ideal solution
clearly is not to neglect any level but to impose constraints at every
level, ranging from the generic at the lowest level to the specific at the highest.  
This is sometimes referred to as multi-level modelling.
Shifts from one set of high-level constraints to
another can then be used to model the transition from one state of the
system to another while preserving integrity through the lower levels.

In this chapter, we will give an overview of a new generalised coarse-grained
multi-level simulation method that we have developed along the lines
described above with a view to simulating the motion of large macromolecular
systems and, in particular, molecular motors.   
