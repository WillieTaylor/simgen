\begin{thebibliography}{100}

\bibitem[1]{S-CohenFEet80b}
F.~E. Cohen, M.~J.~E. Sternberg, and W.~R. Taylor.
\newblock Analysis and prediction of protein $\beta$-sheet structures by a
  combinatorial approach.
\newblock {\em Nature}, 285:378--382, 1980.

\bibitem[2]{S-CohenFEet82a}
F.~E. Cohen, M.~J.~E. Sternberg, and W.~R. Taylor.
\newblock Analysis and prediction of the packing of $\alpha$-helices against a
  $\beta$-sheet in the tertiary structure of globular proteins.
\newblock {\em J. Molec. Biol.}, 156:821--862, 1982.

\bibitem[3]{S-TaylorWRet83a}
W.~R. Taylor and J.~M. Thornton.
\newblock Prediction of super-secondary structure in proteins.
\newblock {\em Nature}, 301:540--542, 1983.

\bibitem[4]{S-TaylorWR86a}
W.~R. Taylor.
\newblock Identification of protein sequence homology by consensus template
  alignment.
\newblock {\em J. Molec. Biol.}, 188:233--258, 1986.

\bibitem[5]{S-PearlLHet87b}
L.~H. Pearl and W.~R. Taylor.
\newblock A structural model for the retroviral proteases.
\newblock {\em Nature}, 329:351--354, 1987.

\bibitem[6]{S-TaylorWR88a}
W.~R. Taylor.
\newblock A flexible method to align large numbers of biological sequences.
\newblock {\em J. Molec. Evol.}, 28:161--169, 1988.

\bibitem[7]{S-TaylorWRet89a}
W.~R. Taylor and C.~A. Orengo.
\newblock Protein structure alignment.
\newblock {\em J. Molec. Biol.}, 208:1--22, 1989.

\bibitem[8]{S-TaylorWR91b}
W.~R. Taylor.
\newblock Towards protein tertiary fold prediction using distance and motif
  constraints.
\newblock {\em Prot. Engng.}, 4:853--870, 1991.

\bibitem[9]{S-JonesDTet92b}
D.~T. Jones, W.~R. Taylor, and J.~M. Thornton.
\newblock A new approach to protein fold recognition.
\newblock {\em Nature}, 358:86--89, 1992.

\bibitem[10]{S-TaylorWR93a}
W.~R. Taylor.
\newblock Protein fold refinement: building models from idealised folds using
  motif constraints and multiple sequence data.
\newblock {\em Prot. Engng.}, 6:593--604, 1993.

\bibitem[11]{S-AszodiAet94a}
A.~Asz\'{o}di and W.~R. Taylor.
\newblock Folding polypeptide $\alpha$-carbon backbones by distance geometry
  methods.
\newblock {\em Biopolymers}, 34:489--506, 1994.

\bibitem[12]{S-TaylorWRet94a}
W.~R. Taylor, D.~T. Jones, and N.~M. Green.
\newblock A method for $\alpha$-helical integral membrane protein fold
  prediction.
\newblock {\em Prot. Struct. Funct. Genet.}, 18:281--294, 1994.

\bibitem[13]{S-TaylorWRet94d}
W.~R. Taylor, T.~P. Flores, and C.~A. Orengo.
\newblock Multiple protein structure alignment.
\newblock {\em Prot. Sci.}, 3:1858--1870, 1994.

\bibitem[14]{S-AszodiAet95c}
A.~Asz\'{o}di, M.~J. Gradwell, and W.~R. Taylor.
\newblock Global fold determination from a small number of distance restraints.
\newblock {\em J. Molec. Biol.}, 251:308--326, 1995.

\bibitem[15]{S-TaylorWR97d}
W.~R. Taylor.
\newblock Multiple sequence threading: an analysis of alignment quality and
  stability.
\newblock {\em J. Molec. Biol.}, 269:902--943, 1997.

\bibitem[16]{S-TaylorWR98a}
W.~R. Taylor.
\newblock Dynamic databank searching with templates and multiple alignment.
\newblock {\em J. Molec. Biol.}, 280:375--406, 1998.
 
%\bibitem[]{S-PollockDDet99a}
%D.~D. Pollock, W.~R. Taylor, and N.~Goldman.
%\newblock Coevolving protein residues: maximum likelihood identification and
%  relationship to structure.
%\newblock {\em J. Molec. Biol.}, 287:187--198, 1999.

\bibitem[17]{S-TaylorWR99b}
W.~R. Taylor.
\newblock Protein structure domain identification.
\newblock {\em Prot. Engng.}, 12:203--216, 1999.

\bibitem[18]{S-TaylorWR00b}
W.~R. Taylor.
\newblock A deeply knotted protein and how it might fold.
\newblock {\em Nature}, 406:916--919, 2000.

\bibitem[19]{S-TaylorWR02a}
W.~R. Taylor.
\newblock A periodic table for protein structure.
\newblock {\em Nature}, 416:657--660, 2002.

\bibitem[20]{TaylorWRet08a}
W.~R. Taylor, G.~J. Bartlett, V.~Chelliah, D.~Klose, K.~Lin, T.~Sheldon, and
  I.~Jonassen.
\newblock Prediction of protein structure from ideal forms.
\newblock {\em Proteins: struct., funct., bioinfo.}, 70:1610--1619, 2008.

\bibitem[20]{T}
\huge \bf Twenty Selected Papers

\end{thebibliography}  
